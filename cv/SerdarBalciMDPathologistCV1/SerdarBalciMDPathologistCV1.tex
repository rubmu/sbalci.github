%!TEX TS-program = xelatex
%!TEX encoding = UTF-8 Unicode
% Awesome CV LaTeX Template for CV/Resume
%
% This template has been downloaded from:
% https://github.com/posquit0/Awesome-CV
%
% Author:
% Claud D. Park <posquit0.bj@gmail.com>
% http://www.posquit0.com
%
% Template license:
% CC BY-SA 4.0 (https://creativecommons.org/licenses/by-sa/4.0/)
%


%-------------------------------------------------------------------------------
% CONFIGURATIONS
%-------------------------------------------------------------------------------
% A4 paper size by default, use 'letterpaper' for US letter
\documentclass[11pt, a4paper]{awesome-cv}

% Configure page margins with geometry
\geometry{left=1.4cm, top=.8cm, right=1.4cm, bottom=1.8cm, footskip=.5cm}

% Specify the location of the included fonts
\fontdir[fonts/]

% Color for highlights
% Awesome Colors: awesome-emerald, awesome-skyblue, awesome-red, awesome-pink, awesome-orange
%                 awesome-nephritis, awesome-concrete, awesome-darknight
\colorlet{awesome}{awesome-red}
% Uncomment if you would like to specify your own color
% \definecolor{awesome}{HTML}{CA63A8}

% Colors for text
% Uncomment if you would like to specify your own color
% \definecolor{darktext}{HTML}{414141}
% \definecolor{text}{HTML}{333333}
% \definecolor{graytext}{HTML}{5D5D5D}
% \definecolor{lighttext}{HTML}{999999}

% Set false if you don't want to highlight section with awesome color
\setbool{acvSectionColorHighlight}{true}

% If you would like to change the social information separator from a pipe (|) to something else
\renewcommand{\acvHeaderSocialSep}{\quad\textbar\quad}


%-------------------------------------------------------------------------------
%	PERSONAL INFORMATION
%	Comment any of the lines below if they are not required
%-------------------------------------------------------------------------------
% Available options: circle|rectangle,edge/noedge,left/right

\name{Serdar Balci}{}

\position{Associate Professor of Pathology}
\address{Independent Researcher}

\mobile{+90 533 4472096}
\email{\href{mailto:drserdarbalci@gmail.com}{\nolinkurl{drserdarbalci@gmail.com}}}
\homepage{serdarbalci.com}
\github{sbalci}
\linkedin{serdar-balci-md-pathologist}
\twitter{serdarbalci}

% \gitlab{gitlab-id}
% \stackoverflow{SO-id}{SO-name}
% \skype{skype-id}
% \reddit{reddit-id}



% Templates for detailed entries
% Arguments: what when with where why
\usepackage{etoolbox}
\def\detaileditem#1#2#3#4#5{
\cventry{#1}{#3}{#4}{#2}{\ifx#5\empty\else{\begin{minipage}{0.7\textwidth}\begin{cvitems}#5\end{cvitems}\end{minipage}}\fi}}
\def\detailedsection#1{\begin{cventries}#1\end{cventries}}

% Templates for brief entries
% Arguments: what when with
\def\briefitem#1#2#3#4#5{\cvhonor{}{#1}{#3}{#2}}
\def\briefsection#1{\begin{cvhonors}#1\end{cvhonors}}

\providecommand{\tightlist}{%
	\setlength{\itemsep}{0pt}\setlength{\parskip}{0pt}}

%-------------------------------------------------------------------------------
\begin{document}

% Print the header with above personal informations
% Give optional argument to change alignment(C: center, L: left, R: right)
\makecvheader

% Print the footer with 3 arguments(<left>, <center>, <right>)
% Leave any of these blank if they are not needed
\makecvfooter
  {November 2018}
  {Serdar Balci~~~·~~~Curriculum Vitae}
  {\thepage}


%-------------------------------------------------------------------------------
%	CV/RESUME CONTENT
%	Each section is imported separately, open each file in turn to modify content
%-------------------------------------------------------------------------------

\hypertarget{some-stuff-about-me}{%
\section{Some stuff about me}\label{some-stuff-about-me}}

\begin{itemize}
\tightlist
\item
  \url{https://about.me/serdarbalci}
\item
  I use \url{https://twitter.com/serdarbalci} frequently.
\item
  See \url{https://www.serdarbalci.com/} for main web page
\item
  \url{http://www.patolojinotlari.com/} for Turkish Lecture Notes \&
  Social Media Based Notes
\item
  \url{https://sbalci.github.io/MyRCodesForDataAnalysis/} for R codes,
\item
  \url{https://sbalci.github.io/pubmed/BibliographicStudies.html} for
  Bibliographic studies.
\end{itemize}

%-------------------------------------------------------------------------------
\end{document}
